\documentclass[a4paper,11pt]{article}

\usepackage{StealthNet}

% Graphics directory
\ifpdf
    % pdflatex requires bitmap images.
    \graphicspath{{./img/png/}}
\else
    % latex requires vector images.
    \graphicspath{{./img/eps/}}
\fi

% Title page details
\title{StealthNet Security Implementation: \\ Project 1}
\author{Joshua Spence \\ 308216350 \and Ahmad Al Mutawa \\ 312003919 \and James Moutafidis \\ 420105464}
\date{May 2012}

%%%%%%%%%%%%%%%%%%%%%%%%%%%%%%%%%%%%%%%%%%%%%%%%%%%%%%%%%%%%%%%%%%%%%%%%%%%%%%%%
\begin{document}

\maketitle

% INTRODUCTION
\section{Introduction}
% TODO

% ASYMMETRIC ENCRYPTION
\section{Asymmetric Encryption}

\subsection{Introduction}
Asymmetric encryption refers to an encryption system in which two separate keys 
are used --- namely, a \emph{public key} and a \emph{private key}. The public 
key is used for encryption \textbf{only} and the private key is used for
decryption \textbf{only}. The keys are so named because the public key may be 
made public --- it doesn't matter who has knowledge of this key because all that
this key allows is the encryption of messages, not the decryption of messages. 
Similarly, and more importantly, the private key must be kept private.

Although unrelated, the two parts of the key pair are mathematically linked. The
public key is used to transform a message into an unreadable form, decryptable 
only by using the (different but matching) private key. By publishing the public
key, the key producer empowers anyone who gets a copy of the public key to 
produce messages only they can read --- because only the key producer has a copy
of the private key (required for decryption). When someone wants to send a 
secure message to the creator of those keys, the sender encrypts it (i.e., 
transforms it into an unreadable form) using the intended recipient's public 
key; to decrypt the message, the recipient uses the private key. No one else, 
including the sender, can do so.

Thus, unlike symmetric key algorithms, a public key algorithm does not require a
secure initial exchange of one, or more, secret keys between the sender and 
receiver. These algorithms work in such a way that, while it is easy for the 
intended recipient to generate the public and private keys and to decrypt the 
message using the private key, and while it is easy for the sender to encrypt 
the message using the public key, it is extremely difficult for anyone to figure
out the private key based on their knowledge of the public key. They are based 
on mathematical relationships (the most notable ones being the integer 
factorization and discrete logarithm problems) that have no efficient solution.

\subsection{Details}
\algorithm{RSA} is an algorithm for public-key cryptography that is based on the
presumed difficulty of factoring large integers, the factoring problem. A user 
of \algorithm{RSA} creates and then publishes the product of two large prime 
numbers, along with an auxiliary value, as their public key. The prime factors 
must be kept secret. Anyone can use the public key to encrypt a message, but 
with currently published methods, if the public key is large enough, only 
someone with knowledge of the prime factors can feasibly decode the message. 

% TODO

\subsection{Justification}
% TODO

\subsection{Implementation}
Asymmetric encryption is implemented in \packageName{} on top of the 
\verb+Encryption+ and \verb+AsymmetricEncryption+ base classes. In particular, 
the \verb+RSAAsymmetricEncryption+ class implements the \algorithm{RSA} 
algorithm.

The \verb+Encryption+ base class provides methods \verb+encrypt(byte[])+ and 
\verb+decrypt(byte[])+ for encryption with the peer's public key and decryption 
with the local private key, respectively.  The \verb+AsymmetricEncryption+ base 
class requires a public-private key pair for instantiation, but setting of the 
peer's public key (for encryption) can be delayed until a later stage.

Packed into the JAR file of the StealthNet executable is the public key of the 
\serviceName{} server and the \serviceName{} bank. By reading these public keys 
from the JAR archive, a \serviceName{} is able to make a secure connection to 
the \serviceName{} server, as well as the \serviceName{} bank, without needing 
to worry about the possibility of a man-in-the-middle attack. If an attacker 
does attempt such an attack, they would be unable to decrypt the transmitted 
packets.

Whilst the clients are aware of the server and bank public keys, the server and 
the bank are unaware of the public keys of all of the clients. For this reason, 
when a client first creates a connection to a \serviceName{} server or bank, the
first (encryption) data that is sent is the client's public key. Once the server
has received this public key, the server can itself encrypt client-bound 
messages.

\algorithm{RSA encryption} is initialised in the \verb+initiateSession()+ 
function, after a communications socket has been established. The initiator of 
the connection (usually the client), sends their public key to the other party 
(usually but not always a server) in encrypted form, but only if the other party
does not already have this public key (maintained with the 
\verb+peerHasPublicKey+ variable. Once the public key has been transmitted, the 
\serviceName{} peer calls the \verb+recvPublicKey()+ function, rejecting all 
incoming packets until the peer receives the initiating party's public key. 
Similarly to the initiating party, the party that is accepting the connection 
only waits for the peer's public key value if it does already possess this value
(maintained through the condition 
\verb+symmetricEncryptionProvider.getPeerPublicKey() == null+).

In situations in which a client is connecting to a \serviceName{} server or 
bank, the client will already possess the public key of the server/bank, but the
server/bank will not possess the public key of the client. In situations in 
which a client is connecting to another client, the server informs both parties 
of the public key such that both parties can perform asymmetric encryption from 
the commencement of the communication.

\algorithm{RSA encryption} is implemented as a feature of the \serviceName{} 
\verb+Comms+ class, however all \verb+Comms+ for any given \serviceName{} peer 
will use identical public-private keys, it is only the peer public keys that 
will differ.

The size selected for the public and private keys was selected to be 2048-bits.
 
% FILE ENCRYPTION
\section{File Encryption}

\subsection{Introduction}
File encryption is required within \packageName{} to protect the private keys of
\serviceName{} peers from being stolen by an attacker and consequently used by 
the attacker to masquerade as a legitimate user. To achieve this, \packageName{}
uses password-based encryption to encrypt the private key data before it is 
written to disk. Password-based file encryption differs from \packageName{} 
packet encryption in that the key (password) for file encryption has a much 
longer lifetime than the key used for packet encryption and decryption. The key 
used for password-based encryption is likely to change infrequently, if at all, 
because the user is required to remember the password in order to decrypt the 
file. The keys used in packet encryption, however, are usually only valid for 
the duration of a session and are regenerated for the next session.

\subsection{Details}
Password-based file encryption within \packageName{} is implemented using 
``JCE'' (Java Crypto Extensions), using features provided by ``BouncyCastle''
--- an open source collection of encryption APIs. The particular algorithm used 
for password-based encryption is \verb+PBEWithSHAAnd3KeyTripleDES+ from the 
``BouncyCastle'' API.

\subsection{Justification}
% TODO

\subsection{Implementation}
% TODO

% ASYMMETRIC VERIFICATION
\section{Asymmetric Verification}

\subsection{Introduction}
% TODO

\subsection{Details}
% TODO

\subsection{Justification}
% TODO

\subsection{Implementation}
% TODO

% PAYMENT PROTOCOL
\section{Payment Protocol}

\subsection{Introduction}
% TODO

\subsection{Details}
% TODO

\subsection{Justification}
% TODO

\subsection{Implementation}
% TODO

% GENERAL IMPLEMENTATION DETAILS
\section{General Implementation Details}

\subsection{Security Measures}
The implementation of the various security measures implemented in 
\packageName{} are detailed in the previous sections. In order to share
cryptographic keys between parties, modifications were made to the 
\packageName{} \verb+Packet+ class to accommodate three additional command types:
\begin{itemize}
\item \verb+CMD_AUTHENTICATIONKEY+: For exchanging \algorithm{Diffie-Hellman} 
keys between the parties.
\item \verb+CMD_INTEGRITYKEY+: For exchanging \algorithm{SHA-1} keys used for
\algorithm {Hashed Message Authentication Code (HMAC)} generation and 
verification.
\item \verb+CMD_NONCESEED+: A seed for a \verb+PRNGNonceGenerator+, 
which is used to generate and verify nonces that are appended to a packet in 
order to prevent replay prevention.
\end{itemize}

These additional packet commands should not appear during normal execution of 
\serviceName{}, but rather should only exist during execution of the
\verb+initiateSession()+ and \verb+acceptSession()+ functions, and are handled 
completely within the \verb+Comms+ class.

\subsection{Packet}
In order to share additional data between communicating parties, the 
\serviceName{} \verb+Packet+ class was renamed to \verb+DecryptedPacket+ and 
extended such that it is composed of the following fields:
\begin{itemize}
\item \verb+command+: The command type being transmitted in the 
packet.
\item \verb+data+: The command data being transmitted in the packet.
\item \verb+nonce+: The unique nonce, generated by a pseudo-random number 
generator in order to allow a given message to be received only once on the 
receiving end of the communications.
\end{itemize}

An additional class, \verb+EncryptedPacket+ was designed to store the following
data:
\begin{itemize}
\item \verb+data+: The encrypted contents of the corresponding 
\verb+DecryptedPacket+.
\item \verb+digest+: The MAC digest of the encrypted packet contents, used to 
verify packet integrity.
\end{itemize}

\subsection{Order of Application of Security Measures}
The security measures are applied to packet data in the following order:
\begin{enumerate}
\item Nonce generation
\item Encryption
\item MAC digest calculation
\end{enumerate}

This ordering effectively defines the degree to which the various security 
protocols are effective. MAC digest calculation generates a MAC digest based on
the encrypted packet's \verb+command+, \verb+data+ and \verb+nonce+ fields and 
can, consequently, be used to verify the integrity of these three fields. All 
security methods occur internally to the \verb+DecryptedPacket+ and 
\verb+EncryptedPacket+ classes, either at the time that the packet is 
instantiated or just before/after the packet is transmitted/received.

Encryption is applied to all \verb+DecryptedPacket+ fields --- \verb+command+, 
\verb+data+ and \verb+nonce+. Whilst not strictly necessary, this ensures that 
all of these fields remain private from a potential attacker, giving the 
attacker very little information on which to base an attack. 

\subsection{Proxy}
In addition to the provided \serviceName{} \verb+Client+ and \verb+Server+ 
classes, I have implemented an additional \serviceName{} entity --- a 
\serviceName{} \verb+Proxy+. This class acts as a ``man-in-the-middle'' for
\serviceName{} communications. In its normal mode of operation, this class will
simply accept incoming \serviceName{} communications as if it were a 
\serviceName{} server, and will then create its own additional \serviceName{}
communication to the \emph{real} \serviceName{} server. Once these channels have
been created, the proxy simply forwards all received packet strings (note that 
the proxy makes no attempt to parse strings into \verb+Packet+s) to the other
party. So, in its normal mode of operation, using \serviceName{} through the
proxy will be transparent to both the client and the server, and will have no
impact on the quality of the communications.

However, the \verb+Proxy+ class has additional functionality to simulate 
various security attacks. The proxy server is, for example, able to replay 
transmitted packets and modify the encrypted packet data. The usefulness of this
class is that it can provide these simulations, thus allowing the effectiveness
of the implementation of the security protocols in \serviceName{} to be 
observed.

% CONCLUSION
\section{Conclusion}
% TODO

\end{document}
