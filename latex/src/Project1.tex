\documentclass[a4paper,11pt]{article}

\usepackage{StealthNet}

% Graphics directory
\ifpdf
    % pdflatex requires bitmap images.
    \graphicspath{{./img/png/}}
\else
    % latex requires vector images.
    \graphicspath{{./img/eps/}}
\fi

% Title page details
\title{StealthNet Security Implementation: \\ Project 1}
\author{Joshua Spence \\ 308216350}
\date{April 2012}

%%%%%%%%%%%%%%%%%%%%%%%%%%%%%%%%%%%%%%%%%%%%%%%%%%%%%%%%%%%%%%%%%%%%%%%%%%%%%%%%
\begin{document}

\maketitle

% INTRODUCTION
\section{Introduction}
This document outlines the modifications that were made to the \packageName{} in
order to secure the communications provided by the application. The provided
source code for the \packageName{} package is vulnerable to a wide range of 
\singleQuote{attacks}, including but not limited to the following.

\begin{description}

\item[Eavesdropping] The communications implemented in \packageName{} transmit
messages in cleartext over network sockets. This allows an attacker who has 
gained access to data paths in the network to eavesdrop on \packageName{} 
communications.

\item[Data Modification] As a follow-on from \attackName{eavesdropping}, an 
attacker would additionally be able to alter the transmitted data. An attacker
can modify the data in the packet without the knowledge of the sender or 
receiver.

\item[Denial-of-Service] A \attackName{denial-of-service} attack prevents normal
of the \serviceName{} service. An attacker could overwhelm the \serviceName{} 
server such that it is unable to process legitimate connections from legitimate 
users. Alternatively, an attacker could simply block network traffic completely,
resulting in a complete lack of communication between \serviceName{} peers.

\item[Man-in-the-Middle] A \attackName{man-in-the-middle} attack occurs when 
someone between two \serviceName{} peers is actively monitoring, capturing, and 
controlling the communications transparently. An attacker makes independent 
connections with the victims and relays messages between them, making them 
believe that they are talking directly to each other over a private connection.

\end{description}

% AUTHENTICATION
\section{Authentication}

\subsection{Introduction}
Authentication is the process of determining whether someone or something is, in
fact, who or what it is declared to be. In computer networks, authentication is 
an important procedure in which clients can verify a server's identity, and 
servers can similarly verify a client's identity. A successful authentication 
process could prevent an attacker from masquerading as a legitimate user,
illegitimately sending and receiving data on behalf of that user.

\subsection{Details}
The \algorithm{Diffie-Hellman key exchange} method allows two parties that have 
no prior knowledge of each other to jointly establish a shared secret key over 
an insecure communications channel. \algorithm{Diffie-Hellman key exchange} 
involves combining one party's `private key' with the other party's `public key'
in order to generate a number. The other party does the same with its own 
private key and the original party's public key. Through the magic of 
\algorithm{Diffie-Hellman}, both parties are able to generate the same shared 
secret key.

The protocol depends on the discrete logarithm problem for its security. It 
assumes that it is computationally infeasible to calculate the shared secret key
$k = generator^{ab} \bmod prime$ given the two public values 
$generator^{a} \bmod prime$ and $generator^{b} \bmod prime$ when the prime is 
sufficiently large. Maurer has shown that breaking the 
\algorithm{Diffie-Hellman protocol} is equivalent to computing discrete 
logarithms under certain assumptions.

\subsection{Justification}
The \algorithm{Diffie-Hellman key exchange} method was chosen because it is one
of the most common protocols used in networking today, such as in \verb+SSL+ 
(Secure Socket Layer), \verb+TLS+ (Transport Layer Security) and \verb+SSH+
(Secure SHell).

\algorithm{Diffie-Hellman} has the advantage of using asymmetric cryptographic 
keys --- also commonly referred to as ``public'' keys. Key management for an
asymmetric cryptographic key is a lot easier than key management for a
symmetric cryptographic key. However, asymmetric encryption/decryption is many 
times slower than symmetric encryption/decryption, but this does not present any 
great issue in the \packageName{} implementation because the 
\algorithm{Diffie-Hellman key exchange} is used only to exchange symmetric
encryption keys. Our \algorithm{Diffie-Hellman} implementation uses 1024-bit 
keys, which is fairly common and is adequate for the \serviceName{} system.

\subsection{Implementation}
Authentication is implemented in \packageName{} with the 
\verb+KeyExchange+ interface. In particular, the \verb+DiffieHellmanKeyExchange+
class implements the \verb+KeyExchange+ interface.

The \verb+KeyExchange+ interface provides a method for getting the public key
value, as well as a method to generate the shared secret. The actual 
\algorithm{Diffie-Hellman} algorithm is implemented in the 
\verb+DiffieHellmanKeyExchange+ class, which either generates or is passed a 
large prime number and a generator in order to initialise the 
\algorithm{Diffie-Hellman} public-private key pair. In our implementation of 
\algorithm{Diffie-Hellman key exchange}, 1024-bit keys are used.

\algorithm{Diffie-Hellman key exchange} is initialised in the 
\verb+initiateSession()+ function, after a communications socket has been 
established. The initiator of the connection (usually the client), calls the
\verb+initKeyExchange()+ function, which generates their 
\algorithm{Diffie-Hellman} key pairs and transmits their own public key value to
the peer in a \verb+CMD_AUTHENTICATIONKEY+ packet. Once the public key has been 
transmitted, the \serviceName{} peer calls the \verb+waitForKeyExchange()+ 
function, rejecting all incoming packets until the peer receives the other
party's public key value. On the accepting end of the communications, the peer
does not initiate key exchange but rather calls \verb+waitForKeyExchange()+ 
after establishing the communications socket.

When the \verb+waitForKeyExchange()+ function returns, the StealthNet 
communications class will have had its \verb+authenticationProvider+ and \\
\verb+authenticationKey+ variables set.

\algorithm{Diffie-Hellman key exchange} is implemented as a feature of the
\serviceName{} \verb+Comms+ class, which means that every connection to and from
a host will use a different public-private key pair.
 
% ENCRYPTION
\section{Encryption}

\subsection{Introduction}
Encryption is the process of transforming `plaintext' data using a cipher to 
make it unreadable to anyone except those possessing the key. In this context,
encryption also refers to the reverse process of ``decryption'' --- making
encrypted data readable again. In \packageName{}, we have implemented the
\algorithm{Advanced Encryption Standard (AES)} with a 128-bit key length and a 
block size of 16-bytes (128-bits).

For cryptographers, a cryptographic `break' is anything faster than a brute 
force—performing one trial decryption for each key. Thus, an attack against a 
128-bit-key AES requiring $2^{100}$ operations (compared to $2^{128}$ possible 
keys) would be considered a break.

\subsection{Details}
\algorithm{AES} is a block cipher based on a design principle known as a 
`permutation-substitution network'. It has a fixed block size of 128-bits and a
key size of either 128-, 192- or 256-bits. 
% TO FINISH

\subsection{Implementation}
Today, AES has become the de facto standard for encryption algorithms. The 
United States government has declared that AES with a key size of 128-bits (such
as that implemented in our software) is sufficient to protect classified 
information up to the \textsc{Secret} level, but not for data classified at the
\textsc{Top Secret} level. For this reason, we believe that, given its 
purpose, a 128-bit key size is sufficient for \serviceName{}.

In our implementation, we used the \algorithm{Cipher-Block Chaining (CBC)} mode 
of operation for our cipher. In CBC mode, each block of plaintext is 
\verb+XOR+ed with the previous ciphertext block before being encrypted. This 
way, each ciphertext block is dependent on all plaintext blocks processed up to 
that point. Also, to make each message unique, an initialization vector must be 
used in the first block. \algorithm{CBC} has the advantage over the 
\algorithm{Electronic Code Book (ECB)} mode in that the \verb+XOR+ing process 
hides plaintext patterns. CBC has been the most commonly used mode of operation 
and for this reason, is sufficient for the \serviceName{} system.

In our implementation, we used \algorithm{PKCS5} padding.  A block cipher 
requires input that is an exact multiple of the block size. If the input buffer 
is not a multiple of the block size, it must be padded to fill up the last 
block. In general, a padding scheme is used to ensure that input plaintext is a 
multiple of the input block size, in our case 128-bits. The padding is added 
before encryption and removed after decryption. \algorithm{PKCS5} is one 
algorithm to provide this padding, and is the algorithm chosen for our 
\packageName{} implementation. This algorithm pads even if the length of data is
already a multiple of 8 bytes; in such a case, it adds 8 extra bytes. It always 
pads so that when decrypting, the last byte is guaranteed to be a pad byte, 
namely, a number between 1 and 8 indicating the total number of pad bytes.

% INTEGRITY
\section{Integrity}

\subsection{Introduction}
In information security, integrity means that data cannot be modified 
undetectably. Integrity is violated when a message is actively modified in 
transit. In order to provide integrity, each transmitted packet is appended with
a \algorithm{Message Authentication Code (MAC)}, which can be calculated by the
sender and verified by the receiver. A \algorithm{MAC} is produced using by 
applying a keyed hash function to the packet contents, using a shared secret key
for both the sender and receiver. In this way, an attacker (without knowing the 
key) should be unable the alter the transmitted data in such a way that the 
modified packet matches a modified \algorithm{MAC} digest.

\subsection{Details}
\algorithm{HMAC (Hash-based Message Authentication Code)} is a specific 
construction for calculating a \algorithm{Message Authentication Code (MAC)} 
involving a cryptographic hash function in combination with a secret key. As 
with any \algorithm{MAC}, it may be used to simultaneously verify both the data 
integrity and the authenticity of a message. Any cryptographic hash function, 
such as \algorithm{MD5} or \algorithm{SHA-1}, may be used in the calculation of 
a \algorithm{HMAC}; and the resulting \algorithm{MAC} algorithm is termed 
\algorithm{HMAC-MD5} or \algorithm{HMAC-SHA1} accordingly. The cryptographic 
strength of the \algorithm{HMAC} depends upon the cryptographic strength of the 
underlying hash function, the size of its hash output length in bits, and on the
size and quality of the cryptographic key. In our implementation, we use 
\algorithm{HMAC-SHA1} to provide 20-byte 
\algorithm{Message Authentication Codes}.

\subsection{Justification}
\algorithm{SHA-1} is known to be more secure than \algorithm{MD5}, and thus it
makes more sense to implement a \algorithm{SHA1}-based \algorithm{HMAC} 
algorithm within \packageName{}. In fact, \algorithm{SHA-1} is the most widely 
used of the existing \algorithm{SHA} hash functions.

\subsection{Implementation}
Our implementation of \algorithm{Message Authentication Codes} in \packageName{}
is provided using the \verb+MessageAuthenticationCode+ interface, which provides
prototype methods \verb+createMAC()+ and \verb+verifyMAC()+. In particular, a \\ 
\verb+HashedMessageAuthenticationCode+ class is provided, which implements the
\algorithm{Message Authentication Code} algorithm using a keyed \algorithm{MD5} 
hash function. The key for the hash function is generated by the initiator of 
the communications session (in the \verb+initiateSession()+ method), and is 
shared with the peer that is accepting the communications session through a \\
\verb+CMD_INTEGRITYKEY+ packet. After initialisation of the \\
\verb+HashedMessageAuthenticationCode+ instances, each and every packet that is
created by a peer is appended with a \algorithm{Message Authentication Code}. 
At the receiving end of the communication, the \verb+recvPacket()+ function 
verifies the \algorithm{Message Authentication Code} and discards the packet if
the received packet fails verification.

% REPLAY PREVENTION
\section{Replay prevention}

\subsection{Introduction}
Replay prevention is the process by which an attacker stores some transmitted 
data and chooses to ``replay'' (that is, retransmit) that data at a later time 
in an unaltered form. Since the message has not been modified in any way, the 
server would otherwise be unaware that this message is being replayed. The MAC
digest of the message would remain intact and the message itself would remain
encrypted. And so the party receiving the replayed message may be convinced that
this is a legitimate request from another party, when in fact it is not.

In order to prevent this kind of attack, we need to implement a means of 
``replay prevention''. The implementation of replay prevention in \packageName{}
is detailed in this section.

\subsection{Details}
In order to implement ``replay prevention'' in \packageName{}, we decided to 
append each packet with a `nonce' before it was being transmitted. Each nonce 
should be unique and follow some predictable sequence. In this way, both parties
are able to predict and verify valid nonce numbers. Once the receiving party 
receives a packet, it marks the nonce of that packet as having been received. 
Thus, if an attacker tries to replay this same message at a later time, the 
receiving party is able to detect the replay attack (by way of checking the 
nonce) and discard the packet.

We decided to use a pseudo-random number generator (PRNG) to generate packet 
nonces. So that the nonces would follow a predictable sequence, a seeded PRNG
was used and the seed was shared between both parties. Each party has a PRNG for
transmitting packets (which would correspond with the other parties receiving 
PRNG) and for receiving packets (which would correspond with the other parties 
transmitting PRNG). Both parties maintain a set of received nonces, and this 
set is consulted every time a packet is received, in order to determine its
validity.

\subsection{Justification}
The proposed method should work well, provided that the number of messages 
transmitted is sufficiently smaller than the nonce of possible values for the 
data type used to represent the nonce. If this is not the case, then the set of 
consumed nonces may need to be periodically flushed.

\subsection{Implementation}
Packet nonces are provided by the \verb+NonceGenerator+ interface, which 
provides the \verb+isAllowed()+ function to verify whether a packet with a given
nonce should be accepted or discard; as well as a \verb+getNext()+ function for
generating the nonces. A PRNG nonce generator is implemented in the 
\verb+PRNGNonceGenerator+ class. This class uses a seeded number generator for
nonce generation.

% GENERAL IMPLEMENTATION DETAILS
\section{General Implementation Details}

\subsection{Security Measures}
The implementation of the various security measures implemented in 
\packageName{} are detailed in the previous sections. In order to share
cryptographic keys between parties, modifications were made to the 
\packageName{} \verb+Packet+ class to accommodate three additional command types:
\begin{itemize}
\item \verb+CMD_AUTHENTICATIONKEY+: For exchanging \algorithm{Diffie-Hellman} 
keys between the parties.
\item \verb+CMD_INTEGRITYKEY+: For exchanging \algorithm{SHA-1} keys used for
\algorithm {Hashed Message Authentication Code (HMAC)} generation and 
verification.
\item \verb+CMD_NONCESEED+: A seed for a \verb+PRNGNonceGenerator+, 
which is used to generate and verify nonces that are appended to a packet in 
order to prevent replay prevention.
\end{itemize}

These additional packet commands should not appear during normal execution of 
\serviceName{}, but rather should only exist during execution of the
\verb+initiateSession()+ and \verb+acceptSession()+ functions, and are handled 
completely within the \verb+Comms+ class.

\subsection{Packet}
In order to share additional data between communicating parties, the 
\serviceName{} \verb+Packet+ class was renamed to \verb+DecryptedPacket+ and 
extended such that it is composed of the following fields:
\begin{itemize}
\item \verb+command+: The command type being transmitted in the 
packet.
\item \verb+data+: The command data being transmitted in the packet.
\item \verb+nonce+: The unique nonce, generated by a pseudo-random number 
generator in order to allow a given message to be received only once on the 
receiving end of the communications.
\end{itemize}

An additional class, \verb+EncryptedPacket+ was designed to store the following
data:
\begin{itemize}
\item \verb+data+: The encrypted contents of the corresponding 
\verb+DecryptedPacket+.
\item \verb+digest+: The MAC digest of the encrypted packet contents, used to 
verify packet integrity.
\end{itemize}

\subsection{Order of Application of Security Measures}
The security measures are applied to packet data in the following order:
\begin{enumerate}
\item Nonce generation
\item Encryption
\item MAC digest calculation
\end{enumerate}

This ordering effectively defines the degree to which the various security 
protocols are effective. MAC digest calculation generates a MAC digest based on
the encrypted packet's \verb+command+, \verb+data+ and \verb+nonce+ fields and 
can, consequently, be used to verify the integrity of these three fields. All 
security methods occur internally to the \verb+DecryptedPacket+ and 
\verb+EncryptedPacket+ classes, either at the time that the packet is 
instantiated or just before/after the packet is transmitted/received.

Encryption is applied to all \verb+DecryptedPacket+ fields --- \verb+command+, 
\verb+data+ and \verb+nonce+. Whilst not strictly necessary, this ensures that 
all of these fields remain private from a potential attacker, giving the 
attacker very little information on which to base an attack. 

\subsection{Proxy}
In addition to the provided \serviceName{} \verb+Client+ and \verb+Server+ 
classes, I have implemented an additional \serviceName{} entity --- a 
\serviceName{} \verb+Proxy+. This class acts as a ``man-in-the-middle'' for
\serviceName{} communications. In its normal mode of operation, this class will
simply accept incoming \serviceName{} communications as if it were a 
\serviceName{} server, and will then create its own additional \serviceName{}
communication to the \emph{real} \serviceName{} server. Once these channels have
been created, the proxy simply forwards all received packet strings (note that 
the proxy makes no attempt to parse strings into \verb+Packet+s) to the other
party. So, in its normal mode of operation, using \serviceName{} through the
proxy will be transparent to both the client and the server, and will have no
impact on the quality of the communications.

However, the \verb+Proxy+ class has additional functionality to simulate 
various security attacks. The proxy server is, for example, able to replay 
transmitted packets and modify the encrypted packet data. The usefulness of this
class is that it can provide these simulations, thus allowing the effectiveness
of the implementation of the security protocols in \serviceName{} to be 
observed.

% VULNERABILITIES
\section{Vulnerabilities}
Whilst various security protocols have been implemented in \serviceName{} in 
order to provide authentication, confidentiality, integrity and replay 
prevention; the communications application is is still by no means completely
secure. Of particular mention, the application is still vulnerable to a 
\algorithm{Man-in-the-Middle attack}. It is possible for an attacker to 
masquerade as a \serviceName{} server such that a client creates a connection to
the attacker instead of to the legitimate server. The client still enables all 
of the appropriate security measures, but every single measure applied is 
compromised because the attack knows all of the keys that were used. The 
attacker can then make its own connection to the \emph{real} \serviceName{} 
server, consequently establish a different set of keys used to provide the 
various security methods.

In the situation described above, both the server and client could be 
completely unaware that the attacker is performing a man-in-the-middle attack. 
Whenever the attacker receives a packet from one party, it is able to decrypt 
and decode that packet to its original contents, and then re-encode the packet 
using the other parties keys. Then the attacker is able to transmit the 
re-encrypted packet to the receiving party, who would accept the packet as a
legitimate packet from the sending party.

% CHOSEN PLAINTEXT ATTACK
\section{Chosen Plaintext Attack}
In a \algorithm{Chosen Plaintext Attack} an attacker can choose a plaintext 
message, encrypt it by using the current key and then study the resulting 
ciphertext to infer further information which decreases the effectiveness of the 
encryption scheme. In the worst case, a chosen-plaintext attack could reveal the
scheme's private key. The strength of our \packageName{} implementation is 
dependent on the strength of \algorithm{AES encryption}, as well as certain 
properties of the creation of the \algorithm{AES} cipher, such as using a 
random initialisation vector.

It is possible for an attacker to attempt to brute force the encryption key, 
trying all keys, decrypting the ciphertext and checks that it matches the 
plaintext. This always works for every cipher, and will give you the matching 
key. The problem with this for an attacker, is that with a key size of 128-bits 
the key space is large enough that you need much more time than the remaining 
lifetime of the universe to check a significant portion of all keys.

There are some known attacks on \algorithm{AES} which are faster than a brute
force attack, however these attacks still require an insurmountable amount of 
time to guarantee the encryption key.

% CONCLUSION
\section{Conclusion}
In our implementation of \packageName{} we applied \algorithm{Diffie-Hellman key
exchange} in order to provide authentication and establish and common shared 
key. We applied \algorithm{AES} encryption with a 128-bit block size and a
128-bit key to encrypt and decrypt all transmitted data. We used a SHA-1 
hashed Message Authentication Code to allow communicating parties to verify the
integrity of the transmitted data. And finally we implemented a seed pseudo-random
number generator to define a predictable sequence of nonces that the receiving 
party can use to detect a replay attack. These various security methods were 
implemented with a strength (to cryptanalysis) sufficient for the intended 
communications of the \serviceName{} service.

\end{document}
